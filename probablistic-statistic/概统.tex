\documentclass[a4paper]{ctexart}
\usepackage{geometry}
%\usepackage{graphicx}
\usepackage{bm}
\usepackage{amsmath}
\usepackage{xcolor}
\geometry{hmargin=1.25in,vmargin=1in}
\begin{document}
\section{基础概率}
\subsection{概率公理}
\noindent
1.\textbf{非负性} $\bm P\ge 0$\\
2.\textbf{规范性} 对于必然事件 $\bm P(\bm \Omega)=1$\\
3.\textbf{可列可加性} 对于两两互不相容的事件$\{\bm A_i\}, \bm{A_iA_j}=\emptyset.$
\begin{equation}
    \bm P(\bigcup_{i=1}^{+\infty} \bm A_i)=\sum_{i=1}^{+\infty} \bm {P(A_i)}
\end{equation}
\subsection{公式}
\begin{align*}
     & P(\bigcup A_i)=\sum P(A_i)-\sum P(A_iA_j)+\sum P(A_iA_jA_k)+\ldots+(-1)^{n-1}P(A_1A_2\ldots A_n) \\
     & |\bigcup A_i|=\sum |A_i|-\sum |A_iA_j|+ \sum |A_iA_jA_k| +\ldots +(-1)^{n-1}|A_1A_2\ldots A_n|   \\
     & |A-B|= |A|-|AB|
\end{align*}
\subsection{条件概率}
\begin{equation}
    \bm P (\bm{A|B})=\frac{\bm P(\bm {AB})}{\bm P(\bm B)}
\end{equation}
假设$\bm \Omega$不变,则条件$\bm B$说明$\bm B$成为全空间。概率即为$\bm A$的测度比上$\bm B$的测度。
\subsection{Bayes公式}
设$\{\bm B_i\}$为$\bm \Omega$ 的一个划分
\begin{equation}
    \bm P(\bm {A|\Omega})=\sum \bm P(\bm B_i)\bm P(\bm {A|B_i})
\end{equation}
于是
\begin{equation}
    \bm P(\bm {B_i|A})=\frac{\bm P (\bm {AB_i})}{\bm P(\bm A)}=\frac{\bm P(\bm B_i)\bm P(\bm {A|B_i})}{\displaystyle \sum \bm P(\bm B_i)\bm P(\bm {A|B_i})}
\end{equation}
\subsection{独立性}
设n个事件$\bm {A_1,A_2,\ldots,A_n}$,相互独立要求
%\begin{equation}
\begin{align}
    \bm P(\bm{A_iA_j})            & =\bm P(\bm A_i)\bm P(\bm A_j) \notag              \\
    \bm P(\bm {A_iA_jA_k})        & =\bm P(\bm A_i)\bm P(\bm A_j)\bm P(\bm A_k)\notag \\
    \ldots\ldots                                                                      \\
    \bm P(\bm {A_iA_j\ldots A_n}) & =\prod \bm p(\bm A_i) \notag
\end{align}
{\color{blue}互斥一定不独立;{看成全空间下均匀散点比较好: 不均匀则有的地方"更浓"}}
%\end{equation}
\clearpage
\section{随机变量}
\subsection{随机变量 is 函数}
随机变量是$\bm \Omega$上的实单值函数$\bm X(\bm \omega)$
\subsection{分布函数}
\begin{equation}
    \bm F(x)=\bm P(\bm X\le x),\qquad -\infty<x<+\infty
\end{equation}
性质:
\begin{equation}
    \begin{aligned}
        \bm P(a<x\le b)           & =\bm F(b)-\bm F(a)                        \\
        \bm P(\bm X=x_0)          & =\bm F(x_0)-\bm F(x_0-0)                  \\
        \text{左极限存在,右连续} & \lim_{t\rightarrow x^+} \bm F(t)=\bm F(x)
    \end{aligned}
\end{equation}
\subsection{离散型随机变量}
离散型随机变量的可能取值是{\color{red}可列}个,设可能取值$\bm X=x_k\ (k=1,2,\ldots)$,不妨设$x_1<x_2<\ldots$,称$\bm P(\bm X=x_k)=p_k,\ k=1,2\ldots$为$\bm X$ 的分布律,且有:\\
\begin{equation}
    \begin{aligned}
        p_k\geq 0                  & \qquad k=1,2,\ldots \\
        \sum_{k=1}^{+\infty} p_k=1 &
    \end{aligned}
\end{equation}

\subsubsection{离散型随机变量分布律}
\paragraph{二项分布$\bm X\~{}\bm B(n,p)$} 进行n次成功k次 $\bm P(\bm X=k)=C_n^k p^k(1-p)^{n-k}$
\paragraph{帕斯卡分布} 成功r次恰需要k次,k成功且前k-1成功r-1
\begin{equation} \bm P(\bm X=k)=p C_{k-1}^{r-1} p^{r-1}(1-p)^{k-r}=C_{k-1}^{r-1}p^r(1-p)^{k-r} \end{equation}
\paragraph{Poisson分布\\}
\textbf{定理: } 若 $\lim\limits_{n\rightarrow+\infty} np_n=\lambda>0$,
\begin{equation}\lim\limits_{n\rightarrow +\infty} C_n^k p_n^k(1-p_n)^{n-k}=e^{-\lambda}\frac{\lambda^k}{k!}\end{equation}
\textbf{Poisson分布X\~{}P($\lambda$):}均匀概率密度,$\lambda$,发生次数。
\begin{equation}
    \bm P(\bm X=k)=e^{-\lambda}\frac{\lambda^k}{k!}
\end{equation}
\subsection{连续型随机变量}
设$\bm X$是一随机变量,$\bm F(x)$ 是他的分布函数,若存在一个非负可积函数$f(X)$,使得\\
\begin{equation}
    \bm F(x)=\int_{-\infty}^x f(t)\mathrm{d}t,\qquad -\infty<x<+\infty
\end{equation}
则X是连续型随机变量,f(x)是概率密度函数,且{\color{blue} 积上去不一定可导(似乎一般可导)}:\\
\begin{equation}
    \begin{aligned}
        f(x)                                     & \geq 0                                       \\
        \int_{-\infty}^{+\infty} f(x)\mathrm{d}x & =\bm F(+\infty) =1                           \\
        f(x)=                                    & \bm F'(x) \qquad where\ f(x)\ is\ continuous \\
    \end{aligned}
\end{equation}
\subsubsection{连续性随机变量分布律}
\paragraph{指数分布 $\bm X \~{} \bm E(\lambda)$:} 无记忆性 or 均匀概率密度,相邻两次发生的间隔。
\begin{equation}
    f(x)=\left\{
    \begin{aligned}
        \lambda e^{-\lambda x}, & \qquad x>0     \\
        0\quad,                 & \qquad x\leq 0
    \end{aligned}\right. \qquad(\lambda>0)
\end{equation}
\paragraph{正态分布 $\bm X\~{}\bm N(\mu,\sigma^2)$}
\begin{equation}
    f(x)=\frac{1}{\sqrt{2\pi} \sigma} e^{-\frac{(x-\mu)^2}{2\sigma^2}},\qquad x\in R\\
\end{equation}
参数 $\mu=0,\ \sigma=1$ 的正态分布称为标准正态分布,记为$\bm X^*\~{} \bm N(0,1)$。其概率密度$\varphi(x)=\frac{1}{\sqrt{2\pi}\sigma} e^{-\frac{x^2}{2}}$.分布函数$\bm \varPhi(x)=\frac{1}{\sqrt{2\pi}}\int_{-\infty}^x e^{-\frac{t^2}{2}}\mathrm{d}t$

\section{二维随机变量及分布律}
\subsection{二维随机变量}
对随机试验$E$的样本空间$\Omega$, 若对$\Omega$中任一样本点$\omega$, 存在实数$(X(\omega),Y(\omega))$与之对应,则称$(X,Y)$为二维随机变量

\subsection{(联合)分布函数}
\textbf{定义:}
\begin{equation}
    F(x,y)=P(\{X\leq x\}\cap \{Y\leq y\})=P(X\leq x,Y\leq y)
\end{equation}\par
联合分布函数的性质:
\begin{center}
    \begin {tabular}{l}
    1. $\qquad 0\leq F(x,y)\leq 1$\\
    $F(-\infty,y)=F(x,-\infty)=F(-\infty,-\infty)=0, F(+\infty,+\infty)=1$\\
    \\
    2. 固定其他变量,对某一特定变量是单调不减函数。\\
    3. 固定其他变量,关于某特定变量右连续\\
    \\
    4. 对于任意的 $a,b,c,d$, 满足 $a<b$, $c<d$,有:\\
    $F(b,d)-F(a,d)-F(b,c)+F(a,c)=P(a<X\leq b,c<Y\leq d)\geq 0$\\
    \color{blue} \textbf{NOTE:} 不像一维那样,这里存在只满足前三点不满足第四点的函数(但第四点是用定义推出来的)
    \end{tabular}
\end{center}

\subsection{边缘分布函数}
\textbf{定义}:边缘分布函数为 $F_X(x)$ 和 $F_Y(y)$.\par
性质:$F_X(x)=F(x,+\infty),\ F_Y(y)=F(+\infty,y)$

\subsection{二维离散随机变量}
\textbf{定义:} 样本点可列,$p_{ij}$ 为分布律。$F(x,y)=\sum\limits_{i=-\infty}^x \sum\limits_{j=-\infty}^y p_{ij}$.\\
\centering $P((X,Y)\in D)=\sum\sum\limits_{(i,j)\in D} p_{i,j}$\\

\raggedright
\subsection{二维连续随机变量}
\textbf{定义:} 若存在非负可积函数$f(x,y)$, s.t. $F(x,y)=\displaystyle\int_{-\infty}^x \int_{-\infty}^y f(u,v) \mathrm{d}u\mathrm{d}v$\\
\centering $\displaystyle P((X,Y)\in D)=\iint_{D} f(x,y) \mathrm{d}x\mathrm{d}y$\\

\raggedright
\subsection{条件分布}
离散:$P(X=x_i|Y=y_j)=\displaystyle\frac{p_{i,j}}{p_{\cdot j}}$\\
连续:在\{Y=y\}条件下X的条件概率密度: $f_{X|Y}(x|y)=\displaystyle\frac{f(x,y)}{f_Y(y)}$

\subsection{独立性}
\begin{center}
    $F(x,y)=F_X(x)F_Y(y).$\\
    $p_{i,j}=p_{i,\cdot}p_{\cdot,j}$\\
    $f(x,y)=f_X(x)f_Y(y)$
\end{center}
\subsubsection{独立性的定理}
1.设 X,Y 为相互独立的随机变量,u(x),v(y)为连续函数。则U,V也相互独立\\
2.若 X,Y 在相互独立的定义域上,有 $f(x,y)=r(x)g(y)$,则X,Y独立。

\subsection{二维随机变量函数的分布}
\begin{equation}
    \begin{aligned}
        P(Z=z_k)=\sum_{g(x_{i_k},y_{j_k})=z_k} P(X=x_{i_k},Y=y_{j_k}) \\
        Poisson\text{分布、二项分布可加(X,Y 独立)}                  \\
        F_Z(z)=\iint_{g(x,y)\leq z} f(x,y) \mathrm{d}x\mathrm{d}y     \\
    \end{aligned}
\end{equation}
\subsubsection{卷积,一般线性函数的分布}
{\color{blue} $f(z)\mathrm{d}z=f(z)\mathrm{d}z\wedge R$} (不太对)
\subsubsection{正态分布可加}
\subsubsection{m个独立指数分布的最小 $\min\{X_1,X_2,\ldots, X_m\} \~{} E(m\lambda)$}


\section{随机变量的数字特征}
\subsection{期望}
\subsubsection{定义:绝对收敛}
\subsubsection{公式}
(对于同一个样本空间下,在同样的随机分布中不同的随机变量$X_i$有:)
\begin{tabular}{l}
    线性:$E((\sum A_iX_i+B_i))=\displaystyle\sum A_iEX_i+b_i$, \\
    独立时 $E(\prod X_i)=\displaystyle\prod E(X_i)$             \\
    多维时总可以把联合分布合起来,这样就是同一个分布函数下的变量问题。
\end{tabular}
\subsection{方差}
\subsubsection{定义}
\begin{equation*}
    \begin{aligned}
        \text{方差:}D(X)=        & E[X-EX]^2=E(X^2)-(EX)^2 \\
        \text{标准差:} \sigma_X= & \sqrt{D(X)}             \\
    \end{aligned}
\end{equation*}

{\small 多维则用联合概率分布。}
\subsubsection{性质}
\begin{equation*}
    \begin{aligned}
         & \text{D存在} \Longleftrightarrow E(X^2)<+\infty. \\
         & D(X-Y)=D(X)+D(Y)-2E[(X-EX)(Y-EY)]                \\
         & D(\sum a_i X_i+b)=\sum a_i^2 D(X_i)
    \end{aligned}
\end{equation*}
\subsection{标准化随机变量}
\begin{equation*}
    X^*=\frac{X-E(X)}{\sqrt{D(X)}}\Longrightarrow EX^*=0, DX^*=1
\end{equation*}
\subsection{协方差和相关系数}
\subsubsection{定义}
\begin{equation*}
    \begin{aligned}
        \text{cov}(X,Y) & =E[(X-EX)(Y-EY)]                                                        \\
        \rho_{XY}       & =\frac{E[(X-EX)(Y-EY)]}{\sqrt{D(X)}\sqrt{D(Y)}}=E[(X^*-EX^*)(Y^*-RY^*)]
    \end{aligned}
\end{equation*}
\subsubsection{公式}
\begin{gather*}
    \rho_{XY}=\frac{cov(X,Y)}{\sqrt{D(X)}\sqrt{D(Y)}}=cov(X^*,Y^*)\\
    cov(X,Y)=E(XY)-E(X)E(Y)\\
    D(X)=cov(X,X)
\end{gather*}
\subsubsection{性质}
独立一定不相关,相关一定不独立。\\
不独立不一定相关,不相关不一定独立

线性:
\begin{equation*}
    cov(\sum_i a_iX_i, \sum_j b_jY_j)=\sum\sum a_ib_j cov(X_i,Y_j)
\end{equation*}
Cauchy-Schwarz:
\begin{gather*}
    |cov(X,Y)|\leq \sqrt{D(X)D(Y)}
\end{gather*}
\begin{center}
    {\small
        证 cov(x,y)是内积: 线性,可交换\\
        正定: $cov(x,x)\geq 0$
    }
\end{center}
\subsection{高阶矩}
\paragraph{k阶原点矩} $E(X^k)$
\paragraph{k阶中心矩} $E([X-E(X)]^k)$
\paragraph{X和Y的k+l阶混合原点矩} $E(X^kY^l)$
\paragraph{X和Y的k+l阶混合中心矩} $E([X-EX]^K[Y-EY]^l)$
\subsubsection{协方差矩阵}
\begin{gather}
    c_{ij}=cov(X_i,X_j)\\
    \bm C=\begin{pmatrix}
        c_{ij}
    \end{pmatrix}
\end{gather}
性质:(内积)\\
n维正态分布:
\begin{equation}
    f(x_1,x_2,\cdots,x_n)=\frac{1}{(2\pi)^{n/2}\vert\bm C\vert^{1/2}} e^{-\frac{1}{2}(\bm x-\bm \mu)^T C^{-1} (\bm x-\bm \mu)}
\end{equation}

\clearpage
\section{大数定理和中心极限}
\subsection{Chebyshev Inequality}
\begin{equation*}
    P(\vert X-E(X) \vert\geq \epsilon)\leq \frac{\sigma^2}{\epsilon^2}, \quad \forall \epsilon>0
\end{equation*}
\subsection{依概率收敛}
已知$\{Y_n\}=Y_1,Y_2,\ldots,Y_n$, $X$是一个随机变量。依概率收敛定义为:$\forall \epsilon>0,$
\begin{equation*}
    \lim_{n\rightarrow +\infty} P(\vert Y_n-X\vert\geq \epsilon)=0 \text{ or } \lim_{n\rightarrow +\infty} P(\vert Y_n -X\vert< \epsilon)=1
\end{equation*}
这里$X$一般单值,也可以 $Y_n = X + X_n$,但可能独立不太行

\subsection{大数定律 —— 次数多起来后频率趋于期望}
\subsubsection{Bernoulli 大数定理 —— 二项分布的频率收敛于概率}
\subsubsection{Chebyshev 大数定律}
设 $\{X_n\}$ 两两不相关,方差存在且一致有界,则
\begin{equation*}
    \lim_{n\rightarrow+\infty} P(\vert \frac{1}{n}\sum_{i=1}^n X_i -\frac{1}{n}\sum_{i=1}^n a_i\vert<\epsilon)=1
\end{equation*}
但需要注意普遍来说这里需要方差,极限只要求期望。这不好
\subsubsection{Khintchine 大数定律}
$\{X_n\}$: \textbf{独立同分布 I,I,D}且期望存在,则
\begin{equation*}
    \lim_{n\rightarrow+\infty} P(\vert \frac{1}{n}\sum_{k=1}^n X_k-E(X)\vert\geq \epsilon)=0
\end{equation*}
变化一下, $X^k\rightarrow E(X_k)$
\subsection{中心极限定理}
\subsubsection{独立同分布的中心极限定理}
设$\{X_n\}$为独立同分布的随机变量序列,且他们的期望方差都存在。$E(X_k)=\mu, D(X_k)=\sigma^2.$,则对任意实数$x$,有:
\begin{equation}
    \lim_{n\rightarrow+\infty} P(\frac{\sum_{i=1}^n X_k-n\mu}{\sqrt{n}\sigma}\leq x)=\varPhi(x)
\end{equation}
若记$\sum_{k=1}^n X_k$ 的标准化随机变量为$Y_n=\frac{\sum_{k=1}^n X_k}{\sqrt{n}\sigma}$, 即 $Y_n\~{} N(0,1)$

\subsubsection{De Moivre-Laplace 中心极限定理}
设随机变量 $Y_n\~{} B(n,p))$,$0<p<1,n=1,2$,则对任一实数$x$,有:
\begin{equation}
    \lim_{n\rightarrow+\infty} P(\frac{Y_n-np}{\sqrt{np(1-p)}}\leq x)=\varPhi(x)
\end{equation}
\clearpage
\section{统计}
\subsection{基本概念}
\paragraph{总体:}研究对象的某个(或某些)数量指标的全体称为\textbf{总体}
\paragraph{个体:}总体的每个元素(数量指标)
\paragraph{样本:}总体的部分
\paragraph{样本观测值:}样本值
\paragraph{样本空间:}样本$(X_1,X_2,\cdots,X_n))$的所有可能取值的集合$\chi=\{(x_1,x_2,\cdots,x_n)\}$
\paragraph{简单随机样本:}独立同分布
\subsection{简单随机样本}
\subsubsection{统计量}
设$(X_1,X_2,\ldots,X_n)$为总体$X$的简单随机样本,$g(r_1,r_2,\cdots,r_n)$是一个实值连续函数,且不含除自变量之外的未知参数.\\
称随机变量$g(X_1,X_2,\ldots,X_n)$为\textbf{统计量},$g(x_1,x_2,\ldots,x_n)$为样本值
\subsubsection{常用统计量}
\paragraph{样本均值} $\bar X$,$\bar x$
\paragraph{样本方差} $S^2=\frac{1}{n-1}\sum_{i=1}^n (X_i-\bar X)^2$, $s^2$ {\color{blue} $\bar X$是一个变量,导致不能提出去。}\\
\paragraph{样本标准差} $S=\sqrt{S^2}$, $s$
\paragraph{样本k阶原点矩} $M_k=\frac{1}{n} \sum_{i=1}^n X_i^k$,$m_k$
\paragraph{样本k阶中心矩} $(CM)_k=\frac{1}{n}\sum_{i=1}^n (X_i-\bar X)^k$, $(cm)_k$
\paragraph{顺序统计量} sort($(X_1,X_2,\cdots, X_n)$),极差
\paragraph{\color{blue}NOTE!! $X$(or $\bar X$)不是一个数,而是一个随机变量(分布)}
$E(\bar X)=\mu, \bar X\neq E(X), D(\bar X)=\frac{\sigma^2}{n}, E(S^2)=\sigma^2$
\paragraph{分位数}
\begin{tabular}{l}
    对于连续随机变量$X$,给定$0<\alpha<1$.                                                                            \\
    若$P(X>x_\alpha)=\alpha$,则$x_\alpha$为$X$所服从分布的上侧$\alpha$分位数。                                        \\
    若$f(x)$为偶函数,则如果$P(\vert X\vert> x_{\alpha/2})=\alpha$,则$x_{\alpha/2}$为$X$服从分布的双侧$\alpha$分位数 \\
    $x_{\alpha/2}=x_{1-\alpha/2}$
\end{tabular}
\subsection{常用分布}
\paragraph{正态分布}
\paragraph{$\chi^2$分布}
设$\{X_n\}$独立且$X_i\~{}N(0,1)$,则称统计量{\color{blue}$\displaystyle\chi^2=\sum_{i=1}^n X_i^2$}: 服从自由度为n的$\chi^2$分布,记为 $\sum_{i=1}^n X_i\~{} \chi^2(n)$.
\begin{equation}
    f_{\chi^2}(x)=\left\{\begin{aligned}
         & \displaystyle\frac{1}{2^{\textstyle\frac{n}{2}}\Gamma(\frac{n}{2})} e^{-\textstyle\frac{x}{2}}x^{\textstyle\frac{n}{2}-1}, & x>0     \\
         & 0,                                                                                                                         & x\leq 0
    \end{aligned}\right.
\end{equation}
其中$\Gamma(x)=\displaystyle\int_0^{+\infty} t^{x-1}e^{-t}\mathrm{d}t$\\
\paragraph{性质:}
\begin{tabular}{l}
    1. $E(\chi^2)=n,\ D(\chi^2)=2n$                                                                  \\
    2. $X_1\~{} \chi^2(n_1),\ X_2\~{} \chi^2(n_2)$, 且$X_1,X_2$独立,则$X_1+X_2\~{} \chi^2(n_1+n_2)$ \\
    3. n 很大时, $\chi^2$近似服从$N(n,2n)$.
\end{tabular}

\paragraph{t分布}
$X\~{} N(0,1), Y\~{} \chi^2(n)$ 且 $X,Y$独立,则{\color{blue}$\displaystyle T=\frac{X}{\sqrt{Y/n}}=\frac{X}{\sqrt{\chi^2(n)/n}}$}: 服从自由度为n的t分布, 记为$T\~{} t(n)$.
\begin{equation}
    f(t)=\frac{\Gamma(\frac{n+1}{2})}{\sqrt{n\pi}\Gamma(\frac{n}{2})}(1+\frac{t^2}{n})^{-\frac{n+1}{2}}, \qquad t\in R
\end{equation}
\paragraph{性质}
\begin{tabular}{l}
    1. $f(t)=-f(t)$                                                                                                       \\
    2. $n\longrightarrow +\infty$, $f(t)\longrightarrow \varphi (t)=\displaystyle\frac{1}{\sqrt{2\pi}}e^{-\frac{t^2}{2}}$ \\
    3. 上侧$\alpha$ 分位数 $t_\alpha(n)$ : $t_{1-\alpha}(n)=-t_\alpha(n)$
\end{tabular}

\paragraph{$F$分布}
设$U\~{}\chi^2(m),\ V\~{}\chi^2(n)$, 且$U$与$V$独立,则称随机变量{\color{blue}$\displaystyle F=\frac{U/m}{V/n}=\frac{\chi^2(m)/m}{\chi^2(n)/n}$}: 服从第一自由度为$m$,第二自由度为$n$的$F$分布,记为 $F\~{} F(m,n)$
\begin{equation}
    f_F(t)=\left\{
    \begin{aligned}
         & \frac{\Gamma(\frac{m+n}{2})}{\Gamma(\frac{m}{2})\Gamma(\frac{n}{2})} (\frac{m}{n})^{\frac{m}{2}}t^{\frac{m}{2}-1}(1+\frac{m}{n}t)^{-\frac{m+n}{2}} & t>0,    \\
         & 0                                                                                                                                                  & t\leq 0
    \end{aligned}
    \right.
\end{equation}
\paragraph{性质}
\begin{tabular}{l}
    1. $F\~{} F(m,n),\ \frac{1}{F}\~{} F(n,m)$ \\
    2. $F(m,n)$的上侧$\alpha$分位数$F_\alpha(m,n)$ : $F_{1-\alpha}(m,n)=\frac{1}{F_\alpha(n,m)}$
\end{tabular}



\clearpage
\subsection{正态总体的抽样分布}
$X\~{} N(\mu,\sigma^2),\{X_i\}\~{} N(\mu,\sigma^2)$是来自总体的$i.i.d$.独立同分布。\\
\begin{align}
    \bar X=                   & \frac{1}{n}\sum_{i=1}^n X_i \~{} N(\mu, \sigma^2/n)  \\
    \frac{(n-1)S^2}{\sigma^2} & =\sum\frac{(X_i-\bar X)^2}{\sigma^2}\~{} \chi^2(n-1) \\
    \bar X \text{与}          & \frac{(n-1)S^2}{\sigma^2} \text{独立}
\end{align}
\begin{align}
    \frac{\bar X-\mu}{\sigma/\sqrt{n}}\~{} N(0,1) \\
    \frac{\bar x-\mu}{S/\sqrt{n}}\~{} t(n-1)
\end{align}\\

\subsubsection*{两个正态总体的情形}
$\{X_n\}\~{} N(\mu_1,\sigma_1^2),\{Y_m\} \~{} N(\mu_2,\sigma_2^2)$. 且$X,Y$独立
\begin{align}
    \frac{S_1^2/\sigma_1^2}{S_2^2/\sigma_2^2}\~{} F(n-1,m-1).
\end{align}
若$\sigma_1=\sigma_2=\sigma$:
\begin{align}
    \bar X-\bar Y\~{} N(\mu_1-\mu_2,\frac{\sigma^2}{n}+\frac{\sigma^2}{n})                      \\
    \text{由可加性} \frac{(n-1)S_1^2}{\sigma^2}+\frac{(m-1)S^2_2}{\sigma^2}\~{} \chi^2(m+n-2)   \\
    \bar X-\bar Y \text{与} \frac{(n-1)S_1^2}{\sigma^2}+\frac{(m-1)S_2^2}{\sigma^2} \text{独立} \\
    \frac{(\bar X-\bar Y)-(\mu_1-\mu_2)}{\sqrt{\frac{1}{n}+\frac{1}{m}}\sqrt{\frac{(n-1)S_1^2+(m-1)S_2^2}{m+n-2}}} \~{} T(m+n-2)
\end{align}
\section{参数估计}
\subsection{点估计}
\subsubsection{频率替代法————大数定律}
\subsubsection{矩估计法}
设总体$X$的分布函数为$F(x;\theta_1,\cdots,\theta_k)$,并假设$k$阶原点矩存在。记
\begin{equation}
    E(X^r)=\mu_r(\bm\theta)\quad r=1,2\ldots,k
\end{equation}
则$\bm \theta$有k个未知数,需要k个方程。从理论上推导出$\bm \mu$的k个表达式,将$\mu_i$替换成样本矩$M^k$,反解$\bm \theta=\hat{\bm\theta}$。称$\hat{\bm \theta}(X_1,\ldots,X_n)$为矩估计量,$\hat{\bm \theta}(\bm x)$为矩估计量
\subsubsection{极大似然估计}
定义似然函数\[L(\bm \theta)=P(\bigcap X_i=x_i)=\prod_{i=1}^n P(x_i;\bm \theta)\]
连续情形下\[L(\bm \theta)=\prod_{i=1}^n f(x_i;\bm \theta)\]
$L$极大,$L(\hat{\bm \theta})=\max L(\bm \theta)$. 称$\hat{\bm \theta}(\bm x)$为最大似然估计值,$\hat{\bm \theta}(X_1,\ldots,X_n)$为最大似然估计量
\subsection{估计量的评价标准}
\subsubsection{无偏性}
对于$\hat{\theta}=\hat{\theta}(X_1,\ldots,X_n)$,若
\[E(\hat{\theta})=\theta\]
则称$\hat{\theta}$是$\theta$的无偏估计量,反之,称$\epsilon=E(\hat{\theta})-\theta$为估计量$\hat{\theta}$的偏差
\begin{tabular}{l}
    ·对任意分布$X$,$\bar X$都是总体均值$\mu=E(X)$的无偏估计量, $S^2$是$\sigma^2$的无偏估计量 \\
    ·$E(\hat\theta)=\theta$不一定得到$E(f(\hat{\theta}))=f(\theta)$
\end{tabular}
\subsubsection{有效性}
设$\hat{\theta_1}$和$\hat{\theta_2}$均为参数$\theta$的\textbf{无偏估计量},若\[D(\hat{\theta_1})<D(\hat{\theta_2})\],称$\theta_1$比$\theta_2$有效\\
\paragraph{·Rao-Cramer不等式\\}
设总体$X$为离散型随机变量, $P(X=x;\theta)=P(x;\theta)$. $(X_1,\ldots,X_n)$是简单随机样本
\begin{equation}
    D(\hat{\theta})\geq I(\theta)=\frac{1}{nE[\displaystyle(\frac{\partial \ln P(X;\theta)}{\partial \theta})^2]}
\end{equation}
设总体$X$为连续性随机变量,概率密度为$f(x;\theta)$,$\hat{\theta}$是\textbf{无偏估计量},
\begin{equation}
    D(\hat{\theta})\geq I(\theta)=\frac{1}{nE[(\displaystyle\frac{\partial \ln f(X;\theta)}{\partial \theta})^2]}
\end{equation}
$I(\theta)$称为无偏估计的方差下界。
\subsubsection{有效估计量}
设$\hat\theta_0$是$\theta$的无偏估计量,如果在所有$\theta$的无偏估计量$\hat\theta$中均有$D(\hat\theta_0)\leq D(\hat\theta)$.称$\hat \theta_0$是$\theta$的有效估计量
\subsubsection{一致性}
设$\hat\theta_n=\hat\theta(X_1,\ldots,X_n)$是$\theta$的估计量,如果$\{\hat\theta_n\}\xrightarrow{p} \theta$, 则称$\hat{\theta}_n$是$\theta$的一致估计量。\\
·不要求无偏。而无偏+方差$\rightarrow$ 0 $\Longrightarrow$ 一致
\subsection{区间估计}
设总体$X$的分布函数为$F(x;\theta)$,$\{X_n\}$是一个样本,若
\[\forall \alpha, \exists \hat{\theta}_1=\hat{\theta}_1(X_1,\ldots,X_n),\hat{\theta}_2=\hat{\theta}_2(X_1,\ldots,X_n).\ s.t.\ P(\hat{\theta}_1<\theta<\hat{\theta}_2)=1-\alpha\]
成立,则称区间$(\hat{\theta}_1,\hat{\theta}_2)$是$\theta$的置信度为$1-\alpha$的置信区间,$\hat{\theta}_1$和$\hat{\theta}_2$是置信下限和致信上限
\paragraph{枢轴量} $U=U(X_1\ldots X_n;\theta)$ 除$\theta$没有其他未知参数,$U$分布已知且不依赖于未知参数
\end{document}